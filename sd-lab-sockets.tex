% !TeX spellcheck = en_GB
\documentclass{beamer}\mode<presentation>{\usetheme{AMSBolognaFC}}
%\documentclass[presentation]{beamer}\mode<presentation>{\usetheme{AMSCesenaPurpleAndGold}}
%%%%

\usepackage{sd-lab-sockets}
%\usepackage{my-listings}

\newcommand{\labN}{4}
\newcommand{\labGroup}{https://gitlab.com/pika-lab/courses/ds/ay2122}
\newcommand{\labRepo}{\labGroup/lab-\labN}

\title[L\labN{} -- Sockets]{L\labN{} -- TCP Sockets}
%
\subtitle[SD]{Distributed Systems\\\scriptsize Technologies}
%
\author[Ciatto \and Omicini]
{\emph{Giovanni Ciatto} \and Andrea Omicini\\
	\texttt{giovanni.ciatto@unibo.it \and andrea.omicini@unibo.it}}
%
\institute[DISI, Univ. Bologna]
{Dipartimento di Informatica -- Scienza e Ingegneria (DISI)\\\textsc{Alma Mater Studiorum} -- Universit{\`a} di Bologna a Cesena}
%
\date[A.Y. 2021/2022]{Academic Year 2021/2022}

\setbeamercovered{transparent}

\AtBeginSection{
	\begin{frame}[c]{Outline}
		% 		\begin{multicols}{2}
		\tableofcontents[sectionstyle=show/shaded, subsectionstyle=hide/hide, subsubsectionstyle=hide/hide]
		% 		\end{multicols}
	\end{frame}
}

\AtBeginSubsection{
	\begin{frame}[c]{Next in Line\ldots}
		\begin{multicols}{2}
			\tableofcontents[sectionstyle=show/shaded, subsectionstyle=show/shaded, subsubsectionstyle=hide/hide]
		\end{multicols}
	\end{frame}
}

\begin{document}

%\\\\\\\\\\\\\\\\\\\\\
\frame{\titlepage}
%\\\\\\\\\\\\\\\\\\\\\

\section{Overview}

\begin{frame}[c]{Motivation \& Lecture Goals}

Sockets are:
%
\begin{itemize}
	\item an ancient, and stable abstraction for low-level networking

	\vfill

	\item very general design/API, virtually supported by all major programming languages

	\vfill

	\item very didactic: they require a good understanding of DS fundamentals
    %
    \begin{itemize}
        \item[eg] client/server, or local/remote dichotomies
    \end{itemize}

	\vfill

	\item very elementary: higher-level communication abstractions can be built on top of them
	%
	\begin{itemize}
		\item[eg] RPC, HTTP, and virtually any application-level protocol
	\end{itemize}
\end{itemize}

\end{frame}

\begin{frame}[c]{Lab \labN{} Repository on GitLab}

	\begin{itemize}
		\item examples and exercises described in this lecture are provided by means of the following GitLab repository:
		%
		\begin{center}
			\url{\labRepo}
		\end{center}

		\vfill

		\item clone it on your machine using Git
		%
		\begin{itemize}
		    \item[\$] \texttt{git clone \textit{<repo URL>}}
		\end{itemize}

		\vfill

		\item even if a minimal environment simply relying on a text editor + Gradle is sufficient for this lab, we kindly suggest to import the cloned repository into some IDE, e.g. IntelliJ Idea or Eclipse

		\vfill

		\item in order to be able to submit your exercises, please ensure you requested access to the \href{\labGroup}{GitLab group of the course}
	\end{itemize}

\end{frame}

\section{About Sockets}



\section{Check Your Understanding}

\startExercise

\subsection{Sequential Echo Server}

\begin{frame}[c,allowframebreaks]{Exercise \currentExercise{} -- Sequential Echo Server}

    TBD

\end{frame}

\startExercise

\subsection{Concurrent Echo Server}

\begin{frame}[c,allowframebreaks]{Exercise \currentExercise{} -- Concurrent Echo Server}

    TBD

\end{frame}


%===============================================================================
\section*{}
%===============================================================================

%\\\\\\\\\\\\\\\\\\\\\
\frame{\titlepage}
%\\\\\\\\\\\\\\\\\\\\\

%%===============================================================================
%\section*{\refname}
%%===============================================================================
%
%%\\\\\\\\\\\\\\\\\\\\\
%%%%%
%%\begin{frame}[t,allowframebreaks]\scriptsize
%\begin{frame}[c]\footnotesize
%\frametitle{\refname}
%\bibliographystyle{apalike}
%\bibliography{sd-lab-sockets}
%\end{frame}
%%\\\\\\\\\\\\\\\\\\\\\

%%%%%%%%%%%%%%%%%%%%%%%%%%%%%%%%%%%%%%%%%%%%%%%%%%%%%%%%%%%%%%%%%%%%%%%%%%%%%%%
\end{document}
%%%%%%%%%%%%%%%%%%%%%%%%%%%%%%%%%%%%%%%%%%%%%%%%%%%%%%%%%%%%%%%%%%%%%%%%%%%%%%%%

